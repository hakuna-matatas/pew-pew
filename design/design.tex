\documentclass{article}
\author{Caleb Mok, Talha Baig, Alan Gao, Newton Ni}
\title{3110 Final Project Design Document}
\begin{document}
\maketitle

    \section{System Description}

        \subsection{Core Vision}
        A battle royale game in Box-Head style. Instead of shooting zombies, shoot other players, with the last one standing winnning.

        \subsection{Key Features}
        � Different lobbies
        � Realtime backend simulation
        � Ocaml web app
        � Concurrency to deal with HTTP requests
        � Combat with different weapons                   

        \subsection{Narrative Description}
	When a user opens the main site, they're prompted to create a temporary display name. Then they'll be directed to the main lobby, where they can select a game. Players can only see and join games that haven't started yet (i.e. they don't have 4 people). Once they join, they spawn empty-handed at random locations through the map. Various weapons, ammunition, and health packs will drop initially, that players will have to search for and pick up. Different weapons will have different properties, like rate of fire, bullet velocity, and bullet spread. As the game goes on, a "wall of death" that encompasses the map will shrink, forcing players to congregate in the center and increasing tension.

    \section{System Design}
	
        \subsection{HTTP}
       	Our game will be designed as a client/server model, where the client communicates with the server via HTTP requests, using JSON to transfer data. We chose to design our system this way to facilitate multiple players connecting to the same game. As for division of labor, the client is responsible only for rendering the images on the user's screen; the server will handle all changes to the game state/model. The exact API can be found in the Apiary link.\\
	\\
	To ensure that the game is synced for all players, clients will request the game state approximately 30 times a second, although the exact number is still uncertain. 

	\subsection{Client}
        The client module will prompt the user to enter a username and to either start a new lobby or pick a lobby to join. The information will be sent to the server, which will start the game after four users are in a lobby. Once the game is started, the server will continually send updates of what the player sees to the client model. The client module will translate that into an image that the player will see. The player's actions such as moving, picking up items, and shooting, will be sent as an HTTP post request to the server. 
    
    \section{Module Design}

        See interfaces.zip.

    \section{Data}
    The server will store the game model, which will contain the following fields and data:
	\begin{verbatim}
	player:
	{
	  uid: string;
	  inventory: string list;    	//This will be item ids
	  location: int*int;
	  health: int;
	  direction: int OR variant    //N, E, S, W, NE, NW, SE, SW
	}	
	
	item: 
	{
	  iid: string;
	  damage: int;
	  projectile speed: int
	  fire rate: int
	}
	
	state:
	{
	  players: player list;
	  items: item list;
	  elems: string*int*int;
	  world boundaries: int*int;
	}
	\end{verbatim}
    When the server returns information to the client, we only return the fields/values that the client needs to render the GUI.
    
    \section{External Dependencies}

            

    \section{Testing Plan}

\end{document}
